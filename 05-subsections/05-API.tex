\section{System API}\label{sec: 5-API}
To provide an easy way of interacting and controling the system we decided to develop a web API. With this API the different parts that compose the complete UrbanSearch system are easily accessible. During the development of the API we have tried to adhere to best-practices and community standards as described in \cite{apigee}. Appendix \ref{sec: API-appendix} contains a detailed description of the available API routes and their functionality.

\subsection{General Remarks}

All routes in the API start with the "/api/v1" prefix. The routes below will be referred to without this prefix to keep the text concise. The API always returns a 200 status code, the response body also contains a status code which indicates if a request was handled successfully.

\subsection{Flask}
The UrbanSearch back end is written completely in the Python programming language. To be able to accept HTTP requests that leverage the system we needed a framework that works efficient with Python code. For this we chose Flask\footnote{\url{http://flask.pocoo.org/}}, a Python microframework that lets us handle incoming requests making use of the functionalities of the UrbanSearch system.\\
The easy setup and good documentation, combined with the small size and efficiency of Flask made us decide to use Flask as our web framework.

\subsection{Blueprints}