\section{Front-End}
An important part of the UrbanSearch system is the part where the extracted and processed data are visualised and made accessible for the end user. Our goals were to provide the end users with a clear and easy to use interface. Extracted relations should therefore be visualised in such a way that the user can make sense of the information easily. Another desire that was expressed by our client, was the possibility to manipulate the displayed information, in a fast, easy and intuitive way. How we tried to reach these goals is described below.

\subsection{Technical Overview}
In this section we will discuss some of the main technical aspects of the UrbanSearch project. We will give an overview of and a motivation for our most important design choices.

\subsubsection{Modular Design}
Dealing with huge amounts of data and displaying this data in a way that makes is easy to understand for users is a challenging task. The complexity of handling this data and making it easy to manipulate by the end-users means an increase in the complexity of our code. Besides this the evolving desires of our client for viewing and manipulating the data lead us to using a modular implementation of the front-end.
Besides the fact that this approach increases readability, maintainability and extensibility it is also a best practice in the front-end realm \footnote{\url{https://developers.google.com/web/fundamentals/}}.\\
Modular development means writing self-contained elements of a web-page, consisting of HTML, CSS and JavaScript. The components can be reused easily throughout the entire page and can be initialised with different sets of data to alter their appearance or functionality.\\
We also used the concept of container and presentational components \footnote{\url{https://medium.com/@dan_abramov/smart-and-dumb-components-7ca2f9a7c7d0}}. The idea behind container and presentational components is that container components are concerned with how things work, e.g. they contain the logic of an application. While presentational components are concerned with how thing looks like, e.g. the styling and appearence of elements.\\
We feel that this approach will result in readable, maintainable and extensible code, which will allow for future proof code.
\subsubsection{NodeJS}

\subsubsection{ExpressJS}


\subsection{Interfaces}
\subsubsection{Interactive Map}
\subsubsection{Classification Interface}
\subsubsection{Settings Interface}

\subsection{Recommendations}