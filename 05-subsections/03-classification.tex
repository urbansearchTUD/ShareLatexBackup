\section{Classification}\label{5-classification}
This section describes how we developed our interface for classifying documents. To provide a complete overview we will first discuss how classifiers are defined. Then we will discuss how classifiers are created or loaded in the system. Finally we will discuss the interface that accepts documents and returns a prediction about the category or categories associated with the supplied document.
\subsection{Scikit Pipelines}
Like we explained in chapter 4 we decided to use the Scikit-Learn library for all our classification logic. A key concept of Scikit is the so called \texttt{Pipeline}. A \texttt{Pipeline} in Scikit is an assembly of intermediate transform steps combined with a final estimator \footnote{\url{http://scikit-learn.org/stable/modules/generated/sklearn.pipeline.Pipeline.html}}. The intermediate transforms transform the input data so that the final estimator can perform optimally. In listing \ref{lst:sdgc} we show an example of an \texttt{Pipeline} that we use in the UrbanSearch system. 

\begin{lstlisting}[language=python, caption={SGDC Pipeline}, label={lst:sdgc}]
sgdc = Pipeline([
            ('tfidf', TfidfVectorizer(stop_words=sw.words('dutch'))),
            ('select', SelectPercentile(f_classif)),
            ('clf', SGDClassifier(alpha=0.0001,
                                  average=False,
                                  class_weight=None,
                                  epsilon=0.1,
                                  eta0=0.0,
                                  fit_intercept=True,
                                  l1_ratio=0.15,
                                  learning_rate='optimal',
                                  loss='log',
                                  n_iter=5,
                                  n_jobs=1,
                                  penalty='l2',
                                  power_t=0.5,
                                  random_state=None,
                                  shuffle=True,
                                  verbose=0,
                                  warm_start=False))
        ])
\end{lstlisting}
This \texttt{Pipeline} consists of three parts. The "tfidf"-part transforms a text in to a matrix of words with associated TF-IDF scores (which are calculated first using the trainig set). The "select"-part selects the best 10 percent of features that were returned by the previous transform, in our case the \texttt{TfidfVectorizer} transform. For this particular \texttt{Pipline} this means that 10 percent of the features with the highest TF-IDF score are returned. Finally the "clf"-part is the final estimator. For this \texttt{Pipeline} it is a SVM that uses SGD training.\\
After defining the pipeline it can be used to train the classifier. This is done by inputting a set of input with corresponding expected outputs. When the classifier is trained it can be used to estimate new unseen inputs. 
\subsection{ModelManagers}

\subsection{ClassifyText Interface}
