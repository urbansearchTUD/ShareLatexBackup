\section{Main Application} \label{sec:main-app}
To couple all functions of the system we made created a main. The main Python file is contains functions which in turn calls all the appropriate functions of the subsystems, thereby providing the functionality that is desired. For example, the \texttt{classify\_textfiles\_to\_db} function (see listing \ref{lst:main-ex}, which combines filtering, classification and storing data while making use of Python multiprocessing to divide and speed up the execution.

\begin{lstlisting}[language=Python, caption=Header of a function in main.py, label={lst:main-ex}]
def classify_textfiles_to_db(num_cworkers, directory, threshold, to_db=False):
    """ Run workers to classify all documents and output to database.
    Database must be online, all the textfiles from the specified directory
    will be parsed using the number of workers specified.
    :num_cworkers: Number of consuming workers, classifying indices from the
    queue.
    :directory: Path to directory containing textfiles
    :to_db: Output results to database specified in config, True or False
    """
\end{lstlisting}

The most important functions in main are listed in table \ref{tbl:main-functions} below. These functions can be called using the API or by adjusting the function that will be called in the \texttt{\_\_name\_\_ == "\_\_main\_\_"}\footnote{\url{https://docs.python.org/3.5/library/__main__.html}} part of main.py, which is the piece of code that will be executed first when executing main.py.


\begin{table}[H]
\centering
\begin{tabular}{m{8cm} m{8cm}}
\textbf{Name}                     & \textbf{Description} \\\hline
\texttt{download\_indices\_for\_url(url)} & Download all indices for a given url and return as string. \\\hline

\texttt{classify\_documents\_from\_indices(pworkers, cworkers, directory, threshold)} & Run workers to classify all documents and only log results, no output to database.\\\hline

\texttt{classify\_indices\_to\_db(pworkers, cworkers, directory, threshold)} & Run workers to classify all documents from a file/directory with indices and output to database.\\\hline

\texttt{classify\_textfiles\_to\_db(num\_cworkers, directory, threshold, to\_db=True)} & Run workers to classify all all textfiles and output to database.\\\hline

\texttt{create\_ic\_relations\_to\_db(num\_workers, to\_db=True)} & Creates intercity relations and stores them in the database if desired. \\\hline 

\end{tabular}
\caption{Functions available in main}
\label{tbl:main-functions}
\end{table}