\subsection{Data-set Route: /datasets}
The datasets route is meant for extending and querying information about the data-set which is used to train classifiers. 

\begin{description}


\item [{\large \textbf{/append}}]
Appends a document to the data-set of the category specified in the request.
\newline
\newline
\textbf{Request:}
\newline
\newline
\begin{tabular}{ | l | l |}
\hline
Method & POST\\ \hline
Content-Type & application/json\\ \hline
\end{tabular}
\newline
\newline
\textbf{Request data:}
\newline
\newline
\resizebox{\textwidth}{!} {
\begin{tabular}{ | l | l | l |}
\hline
\textbf{Property} & \textbf{Required} & \textbf{Description}\\ \hline
document & True & String containing the document that needs to be labelled\\ \hline
category & True & String specifying the category of the data-set we want to append this document to\\ \hline
\end{tabular}}
\newline
\newline
\textbf{Response:}
\newline
\newline
\resizebox{\textwidth}{!} {
\begin{tabular}{ | l | l |}
\hline
\textbf{Property} & \textbf{Description}\\ \hline
status & Status code for the response\\ \hline
error & Boolean indicating if there was an error during the processing of the request\\ \hline
message & Message containing extra information about the response\\ \hline
\end{tabular}}


\item[{\large \textbf{/append\_all}}]
Appends a document to the data-set of all the categories specified in the request.
\newline
\newline
\textbf{Request:}
\newline
\newline
\begin{tabular}{ | l | l |}
\hline
Method & POST\\ \hline
Content-Type & application/json\\ \hline
\end{tabular}
\newline
\newline
\textbf{Request data:}
\newline
\newline
\resizebox{\textwidth}{!} {
\begin{tabular}{ | l | l | l |}
\hline
\textbf{Property} & \textbf{Required} & \textbf{Description}\\ \hline
document & True & String containing the document that needs to be labelled\\ \hline
categories & True & List of strings specifying the categories of the data-sets we want to append this document to\\ \hline
\end{tabular}}
\newline
\newline
\textbf{Response:}
\newline
\newline
\resizebox{\textwidth}{!} {
\begin{tabular}{ | l | l |}
\hline
\textbf{Property} & \textbf{Description}\\ \hline
status & Status code for the response\\ \hline
category & The category that was predicted for this document\\ \hline
error & Boolean indicating if there was an error during the processing of the request\\ \hline
message & Message containing extra information about the response\\ \hline
\end{tabular}}

\item[{\large \textbf{/create}}]
Creates a data-set from all the category specific data-sets.
\newline
\newline
\textbf{Request:}
\newline
\newline
\begin{tabular}{ | l | l |}
\hline
Method & GET\\ \hline
\end{tabular}
\newline
\newline
\textbf{Response:}
\newline
\newline
\resizebox{\textwidth}{!} {
\begin{tabular}{ | l | l |}
\hline
\textbf{Property} & \textbf{Description}\\ \hline
status & Status code for the response\\ \hline
error & Boolean indicating if there was an error during the processing of the request\\ \hline
message & Message containing extra information about the response\\ \hline
\end{tabular}}

\item[{\large \textbf{/create/categoryset}}]
Creates a new file for the category specified in the request. In this file we will save the documents that are submitted for this category 
\newline
\newline
\textbf{Request:}
\newline
\newline
\begin{tabular}{ | l | l |}
\hline
Method & POST\\ \hline
Content-Type & application/json\\ \hline
\end{tabular}
\newline
\newline
\textbf{Request data:}
\newline
\newline
\begin{tabular}{ | l | l | l |}
\hline
\textbf{Property} & \textbf{Required} & \textbf{Description}\\ \hline
category & True & The category for which we want to create a file\\ \hline
\end{tabular}
\newline
\newline
\textbf{Response:}
\newline
\newline
\resizebox{\textwidth}{!} {
\begin{tabular}{ | l | l |}
\hline
\textbf{Property} & \textbf{Description}\\ \hline
status & Status code for the response\\ \hline
error & Boolean indicating if there was an error during the processing of the request\\ \hline
message & Message containing extra information about the response\\ \hline
\end{tabular}}


\item[{\large \textbf{/init\_categorysets}}]
Appends a document to the data-set of all the categories specified in the request.
\newline
\newline
\textbf{Request:}
\newline
\newline
\begin{tabular}{ | l | l |}
\hline
Method & POST\\ \hline
Content-Type & application/json\\ \hline
\end{tabular}
\newline
\newline
\textbf{Response:}
\newline
\newline
\resizebox{\textwidth}{!} {
\begin{tabular}{ | l | l |}
\hline
\textbf{Property} & \textbf{Description}\\ \hline
status & Status code for the response\\ \hline
error & Boolean indicating if there was an error during the processing of the request\\ \hline
message & Message containing extra information about the response\\ \hline
\end{tabular}}


\item[{\large \textbf{/lengths}}]
Returns the lengths of the different category-sets
\newline
\newline
\textbf{Request:}
\newline
\newline
\begin{tabular}{ | l | l |}
\hline
Method & GET\\ \hline
\end{tabular}
\newline
\newline
\textbf{Response:}
\newline
\newline
\resizebox{\textwidth}{!} {
\begin{tabular}{ | l | l |}
\hline
\textbf{Property} & \textbf{Description}\\ \hline
lengths & The lengths of the data-sets per category\\ \hline
status & Status code for the response\\ \hline
error & Boolean indicating if there was an error during the processing of the request\\ \hline
message & Message containing extra information about the response\\ \hline
\end{tabular}}

\end{description}