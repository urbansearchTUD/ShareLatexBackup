\documentclass{article}
\usepackage{hyperref}
\usepackage{titlesec}
\usepackage{enumitem}
\usepackage{fontspec}
\usepackage[a4paper, total={6in, 8.5in}]{geometry}

\setmainfont[
    Path = fonts/,
    Extension = .ttf,
    UprightFont = {*-Regular},
    BoldFont = {*-Bold},
    ItalicFont = {*-Italic},
    BoldItalicFont = {*-BoldItalic},
    Ligatures=TeX, % supposed to replace Mapping=tex-text
    ]{OpenSans}

\begin{document}

\titlespacing{\section}{0pt}{\baselineskip}{0pt}
\titlespacing{\subsection}{0pt}{0pt}{0pt}
\titlespacing{\subsubsection}{0pt}{0pt}{0pt}
\fontsize{11pt}{12pt}
\noindent

\noindent
\section*{General Information}

\noindent\textbf{Project title:} UrbanSearch\\
\textbf{Client's organisation:} OTB - Research for the Built Environment, TU Delft\\
\textbf{Final presentation} July 7th, 2017\\
\textbf{Short description:} Design and implement a system to quantify and visualise intercity relations from open data.

\section*{Description}
According to their web site\footnote{\url{https://otb.tudelft.nl}}, "OTB seeks to make a visible contribution to society by addressing societal challenges in the field of the built environment". However, with the growing amount of information available, they have the need for IT solutions to support their research.

This project was intended to design and implement a system that is able to quantify and visualise intercity relations, using the information from open data. Since there is a lot of data freely available, the challenge was to accurately delimit the boundaries of the project and then convert the information from within these boundaries to something the researchers could work with. Moreover, the students had to focus on both a full-fledged back-end and front-end.

The client was happy to see that the students were able to deliver a viable product. However, some work has to be done before the product can fully satisfy the client's needs. For this, recommendations have been made, in order to provide future developers with pointers to continue on the application.

\section*{Members of the Project Team}
\begin{description}[noitemsep]
\item[Piet van Agtmaal] Mostly interested in back-end design, databases and server management, Piet's focus was on designing, implementing and optimising the Neo4j model and configuring the server for production.
\item[Tom Brunner] Mainly interested in big data handling, Tom's focus was on processing all the textual data retrieved from Common Crawl.
\item[Marko Mali\v{s}] His interests lie with machine learning and unsurprisingly dealt with the classifiers for determining the category a document belongs to.
\item[Gijs Reichert] Developing a robot for Schiphol Airport, Gijs likes to handle the low level application parts. He focused on gathering and parsing data from Common Crawl and parallelising the application.
\end{description}

\subsection*{Client:}
Evert Meijers, Research for the Built Environment Group, TU Delft

\subsection*{Coach:}
Claudia Hauff, Web Information Systems Group, TU Delft

\section*{Contact:}
\noindent Piet van Agtmaal, \href{mailto:P.J.C.vanAgtmaal@student.tudelft.nl}{P.J.C.vanAgtmaal@student.tudelft.nl}\\
Tom Brunner, \href{mailto:T.S.Brunner@student.tudelft.nl}{T.S.Brunner@student.tudelft.nl}\\
Marko Mali\v{s}, \href{mailto:M.Malis@student.tudelft.nl}{M.Malis@student.tudelft.nl}\\
Gijs Reichert, \href{mailto:G.M.W.Reichert@student.tudelft.nl}{G.M.W.Reichert@student.tudelft.nl}\\

\noindent The final report for this project can be found at: \url{http://repository.tudelft.nl}.

\end{document}
