\section {Requirement Analysis}
In this section, we first define the design goals. Next, we present user stories that were created together with the client. Then, we list the requirements which followed from the user stories and which the application should meet. To do so we use the MoSCoW method\cite{clegg1994case} as a prioritisation technique. Lastly, we discuss the design decisions that follow from the design goals and the requirements.

\subsection{Design Goals}
The high-level design goals for this project have been provided by the client. These serve as a guideline to determine the priority label of the specific requirements, as defined in section \ref{sec:reqs} and are listed here, ordered by priority.

\paragraph{credible} The client wants to dispute a widely spread belief. Therefore, the basis on which he does that must be sound.
\paragraph{understandable} The results of the application should be visually understandable, so it is easy for the client to deduce conclusions from them. Additionally, retrievable numeric data enable the client to further investigate the results outside of the scope of the application, should the need arise.
\paragraph{scalable} The client has expressed his concerns that restricting the application to a set of non-English domains might impair the probability that his research will be published in an acknowledged journal. Allowing for investigating other domains would greatly help the client in a later stadium.
\paragraph{pluggable} The client conveyed it might be interesting to let the application perform analysis on different data sets without the need of a developer. 
\paragraph{fast development} Because of the time constraints of the project we need a fast development cycle, this means that choices we make regarding tools, applications and programming languages need to take the time constraint into account.

\subsection{User Stories}
\todo {MMMMARKO?}


\subsection{Product Requirements}\label{sec:reqs}
\todo{TOM TOM TOM TOM TOM}
The MoSCoW method is a prioritisation technique used in management, business analysis, project management, and software development to reach a common understanding with stakeholders on the importance they place on the delivery of each requirement - also known as MoSCoW prioritisation or MoSCoW analysis. Four levels of priority are defined: must have, should have, could have and won't have (also known as would like).
\todo{ add req from user stories}


\subsubsection {Must Have}
Requirements labelled as "must have" are key to the minimal performance of the application If they are not met, the application can be considered a failure.

\begin{enumerate}
    \item A user must be able to input place names.
    \item The system should display a map with the before mentioned places and the important connection they have to other places.
    \item A user must be able to click on a connection between two places and get information about what kind of relations they have.
    \item The strength of all relations must be displayed.
    \item The user must be able to export the found connections and their strengths between places.
\end{enumerate}
 
\subsubsection {Should Have}
"Should have" requirements are those that greatly improve system performance and/or usability but might not fit in the available development time.

\begin{enumerate}
    \item The application should be able to use multiple data sets.
    \item The application should be able to group the relations (e.g. 'fish-trade' and 'finance' to economy, 'medicine' to health-care etc).
    \item The user should be able to 'zoom' on places in order to see more/less connections to other places.
\end{enumerate}

\subsubsection {Could Have}
Requirements labelled as "could have" are useful and should be included in the system if time and resources permit.

\begin{enumerate}
    \item The application could be able to use international names.
    \item There could be a front end for the app.
\end{enumerate}

\subsubsection {Would Like}
"Would like" requirements have been agreed upon as not important to deliver within the current time schedule. However, they can be included in future releases.

\begin{enumerate}
    \item The application would be able to show all connections of all places on the map at the same time.
\end{enumerate}

 
\begin{comment}
\begin{enumerate}
    \item{General} 
    \begin{enumerate}
        \item 
    
        \item A user must be able to input city names.
        \item Relationships that are found in the system must be grouped.
        \item A user must be able to zoom on found relations in order to get more details about the relations.
    \end{enumerate}
    
    \item{Search Engine} 
    \begin{enumerate}
        \item All data from a search engine should be gathered.
        \item From this data only the data containing 2 or more city names should be kept and stored.
    \end{enumerate}
    
    \item{Filtering} 
    \begin{enumerate}
        \item This data should t
        
        \item Logic Filters 
        \item Relations Filters
    \end{enumerate}
    
    \item{Machine Learning} 
    \begin{enumerate}
        \item
    
        \item Types of relations
    \end{enumerate}
    
    \item{Visualisation}
    \begin{enumerate}
        \item
    
        \item Statistics of relations? Query relations
        \item Strength of relations
        \item Types: ML CBS defined
    \end{enumerate}
\end{enumerate}


\subsubsection {Should Have}
"Should have" requirements are those that greatly improve system performance and/or usability but might not fit in the available development time.

\begin{enumerate}
    \item{General}
    \begin{enumerate}
        \item Plugable data sets
    \end{enumerate}
    
    \item{Machine Learning}  
    \begin{enumerate}
        \item Generalising relations, grouping relations
    \end{enumerate}
\end{enumerate}


\subsubsection {Could Have}
Requirements labelled as "could have" are useful and should be included in the system if time and resources permit.

\begin{enumerate}
    \item{General}
    \begin{enumerate}
        \item International city names
    \end{enumerate}
    
    \item{Visualisation}    
    \begin{enumerate}
        \item Front end for the app
    \end{enumerate}
\end{enumerate}

\subsubsection {Would Like}
"Would like" requirements have been agreed upon as not important to deliver within the current time schedule. However, they can be included in future releases.
\end{comment}
