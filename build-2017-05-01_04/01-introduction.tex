\section{Introduction}


%mooier opschrijven
% Common belief is that agglomeration benefits are key to economic growth \cite{porter2000location}. However, it remains unknown whether this economic growth's primary cause is the increase in how well the agglomeration is embedded in the global network. \\

Common belief is that agglomeration benefits are key to economic growth \cite{porter2000location}. However, it remains uncertain whether it is these benefits that primarily cause economic growth, or an increase in the position of the agglomeration in the global urban network. \\

The huge amount of textual data generated online and the numerous historic archives are great sources of information on social and economic behaviours. They constitute the bulk of information flowing among each other. Advanced text mining on newspapers and web pages containing city names would allow for a better understanding of the role of information in shaping urban systems.
Similar to research efforts in other domains, such as financial trade \cite{preis2013quantifying} and sales forecasting \cite{wu2014future}, the idea is to develop search queries that capture urban-urban interactions. These interactions are retrieved from information corpora through the co-occurrence of geographical names in textual data. An example of such a query on the Google search engine\footnote{\url{https://www.google.com}} is \texttt{"Rotterdam + Amsterdam " OR "Amsterdam + Rotterdam"}, which searches for the co-occurrence of Amsterdam and Rotterdam. We thus answer the following question {\color{red} FIXME: correct?}: what approach is best suited for extracting and visualising the global urban network from open data? {\color{red}FIXME: define subquestions?}\\

First, we discuss related work in section 2. Second, we identify the requirements for a solution to the problem and discuss issues that might arise in section 3. Third, we develop a methodology for a framework that satisfies the requirements and tackles the issues in section 4. Last, we conclude in section 5 with the results of our research.

% First, we will look at related work that has been done on data from search queries, discussing the similarities to our research. Next we will further analyse the problem in its four main subsections: the extraction of data from the internet; the filtering and categorising of this data; the development of searchable queries and last the visualisation of the found data. From this problem analysis we will draw the requirement analysis. The requirement analysis is then used to decide upon the framework and tools we are going to use. And last we will give a short conclusion.