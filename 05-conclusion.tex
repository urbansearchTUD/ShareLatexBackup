\section{Conclusion}

First, we discussed related work. We saw that there are currently two methods for analysing the relations between cities. Manually analysing search engine data is very slow and requires a lot of man-hours and looking at the different locations where businesses are located is only interesting for the economic relation and still misses a lot of data.
\\

Second, we identified the requirements for a solution to the problem and discuss issues that might arise. The used the MoSCoW model to describe the importance of the different requirements. The most import must haves we found are being able to input place names, displaying a map with the connection data and being able to extract this data.
\\

Third, we developed a methodology for a framework that satisfies the requirements and tackles the issues. We decided to start by using data from Common Crawl, although we might later extend this to other data sources such as Delpher. After selecting relevant data (data which contains 2 or more city names) we store the data with Neo4j. We then use clustering and classifying machine learning to group the data. First we use this on all data to get the general groups (e.g. economy, health-care, immigration) and then we use this on the data per pair of cities to see what the important connection types for each city are. Then we link these connections to the general groups ('fish' might relate to economy.. etc). To visualise this data we use the graph Neo4j provides.