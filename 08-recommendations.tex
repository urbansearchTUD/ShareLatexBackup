\chapter{Recommendations}\label{sec:recommendations}
In this chapter, we will give some pointers for both the back-end and the front-end as to how the system can be extended and improved.
%For future improvements on this project one might decide upon two ways to do this. The first is to look at the requirements which have not yet been fulfilled and fulfil those. Depending on the needs of the user one might want to implement different things. For extending the application to international places one would want to use data from top-level domains other than .nl. However others might be interested in finding different relations, for which one would need to have other training data.
%If one has more time it might be interesting to instead of using a classification algorithm, to choose a clustering algorithm. \todo{extend}

\todo{ Make recommendations for future version, for extending the back-end and front-end
 Try to mention the requirements here}
 
\section{Extending the Back-End}
In this section, we present several recommendations to improve the back-end.

\subsection{Extending the Data Set}
The input data is now limited to dutch documents only. This limits the amount of available data and also limits the network of cities that can be extracted from this data. In a future version it will be interesting to be able to parse other languages as well, starting with English. This would require some investigation regarding stop-words in English and what primarily causes false positives for semantic association of cities in English documents. Next to that, in a future version of the application it would be interesting to be able to use other data than Common Crawl data. This would require a few modifications to the data storage functions because a unique identifier per stored document is needed. These modifications should lead to an application that is able to use a lot more data to find relations between cities, which in turn should lead to more reliable and credible results.


\subsection{Improving Document Filtering}
Other than the required co-occurrence of cities, there is little noise cancelling in documents. Some documents contain lists of cities and are cancelled out as explained in section \ref{sec:filtering_docs}. However, there might be more intuitive ways for better document filtering. One could for example discard the contents of specific HTML elements, such as forms and input fields. Moreover, some kind of domain blacklist could be constructed to filter out entire (sub)domains that are known to contain mostly false-positives. For example, the client mentioned that many pages of Airbnb were reviews from people of all over the country, but represent no relation between these locations. Another way to extend the document filtering is to have some way to allow for city aliases. By doing so, more documents will pass the filtering stage so more relations can be extracted.

\subsection{Constructing More Advanced Classifiers}
\todo{@Piet - Consult with Marko}
\subsection{Data Set}
\subsection{Parameter Optimisation}
\subsection{Empirical Tests of Classifiers}

\subsection{Upgrading the Server}
\todo{@Piet}
As discussed in section \ref{sec:Discussion - Open Issues}, multiple open issues are due to the server we used, which is not as powerful as we would have liked. For future versions, the server should be upgraded to at least 16GB RAM and preferably at least 8 CPUs, to allow for more efficient multiprocessing and to be able to keep the database in memory for much faster reading. Additionally, an SSD instead of HDD would greatly improve the database speed for both reading and writing. To speed up database access even more, it would help to have a dedicated physical server that only runs the database.

\subsection{Building a Command Line Interface}
The delivered application provides an API and a visual interface, however, there is no command-line interface yet. If a user knows how to program Python it would be possible to call functions using the Python interpreter, but this is far from ideal. A future version could have a CLI (Command-Line Interface) that makes calling parts of the system fairly straightforward. For example, this would enable a user to call just the filtering subsystem or just 
\todo{@Gijs - Configuration on demand, starting single components of the app, extracting statistics, etc.}

\section{Extending the Front-End}
\todo{@Marko}