\chapter{The stuff}
\todo{naming}

\subsection{Accessing the Web Interface}
To access the web interface, you either need to be at a TUDelft location, or use a virtual private network (vpn) connected to the TUDelft network. For information on how to use a vpn, please use the TUDelft website ("https://intranet.tudelft.nl/services/fmvgict-pdc/netwerk/draadloos-netwerk/virtual-private-network-vpn/"). When you have done this, you an use the url "http://citynetworks.bk.tudelft.nl/" to access the urbansearch program. When you enter this url you should have a menu to the left of the page, and a map on the right. After a few seconds of loading, the map on the right will display a couple of cities. 


\subsection{The Menu}
On the left of the screen should be a menu displaying three different sections. The first section displays an option to export the data. The second section to determine what cities should be displayed on the map. The last section is to determine what relations should be taken into account and how these should be weighted.

\subsubsection{Selecting Cities}
\subsubsection{Selecting Relations}
\subsubsection{Exporting Data}

\subsection{The Map}
After a few seconds needed to load the intercity-relations a map should be displayed. This map should contain cities. You can use the scroll wheel of the mouse or the plus and minus buttons in the lower right corner of the map for zooming. When hovering over a city with the mouse the population of that city will be displayed. The cities showing relations will have a slightly darker color than cities that have not been clicked on. To show the relations of a city, that city can be clicked on. You can also click on a relation. This will open up an extra section in the menu containing information about that relation.

\subsubsection{Clicking on relations}
When clicking on a relation a extra section in the menu will be opened. This contains all information about the relation between the two cities. The strength of the relations (amount of documents found for the relation between those cities) will be displayed, as well as the strength of the relations per relation type (such as commuting, shopping, leisure, ..). To check how these relationships were found you can use the option on the lower end of this section to open up another section displaying all documents that connect those two cities.

\subsubsection{Documents connecting two cities menu}
In this new section all documents will be displayed. The only thing that is displayed is the type of relation which these documents fit to and the probability these documents fit that relation are displayed. If a document does not fit to the important relation types (and is thus labelled other), it is not displayed here. You can download the documents in a text file by clicking on them.


