\subsection{API}
To provide the users an easy way of interacting and controling the system we decided to develop an web API. With this API the different parts that compose the complete UrbanSearch system are easily accessible. During the development of the API we have tried to adhere to best-practices and community standards as described in \cite{apigee}. The sections below will describe the parts of the system that are controlled by the API in more detail. Finally we will give some recommendations for the API which we feel would be a good addition/improvement of the API.

\subsubsection{General Remarks}

All routes in the API start with the "/api/v1" prefix. The routes below will be referred to without this prefix to keep the text concise. The API always returns a 200 status code, the response body also contains a status code which indicates if a request was handled successfully.

\subsubsection{Classify Route: /classify}
The classify route is meant as an easy means of labelling a provided document with a category or the probabilities of said document belonging to a set of predefined categories.
The available subroutes are specified below.
\begin{enumerate}
\item[\textbf{/}]
\item[\textbf{/predict}]
Predicts the category of the document that is submitted in the body of the request.
\newline
\newline
\textbf{Request:}
\newline
\newline
\begin{tabular}{ | l | l |}
\hline
Method & POST\\ \hline
Content-Type & application/json\\ \hline
\end{tabular}
\newline
\newline
\textbf{Request data:}
\newline
\newline
\begin{tabular}{ | l | l | l |}
\hline
\textbf{Property} & \textbf{Required} & \textbf{Description}\\ \hline
document & True & String containing the document that needs to be labelled\\ \hline
\end{tabular}
\textbf{Response:}
\newline
\newline
\begin{tabular}{ | l | l |}
\hline
\textbf{Property} & \textbf{Description}\\ \hline
status & Status code for the response\\ \hline
category & The category that was predicted for this document\\ \hline
error & Boolean indicating if there was an error during the processing of the request\\ \hline
message & Message containing extra information about the response\\ \hline
\end{tabular}
\item[\textbf{/probabilities}]
Returns the probabilities of the supplied document belonging to each of the predefined categories. 
\newline
\newline
\textbf{Request:}
\newline
\newline
\begin{tabular}{ | l | l |}
\hline
Method & POST\\ \hline
Content-Type & application/json\\ \hline
\end{tabular}
\\
\textbf{Request data:}\\
\begin{tabular}{ | l | l | l |}
\hline
\textbf{Property} & \textbf{Required} & \textbf{Description}\\ \hline
document & True & String containing the document that needs to be labelled\\ \hline
\end{tabular}
\newline
\newline
\textbf{Response:}
\newline
\newline
\begin{tabular}{ | l | l |}
\hline
\textbf{Property} & \textbf{Description}\\ \hline
status & Status code for the response\\ \hline
probabilities & The probabilities per category that are predicted for this document\\ \hline
error & Boolean indicating if there was an error during the processing of the request\\ \hline
message & Message containing extra information about the response\\ \hline
\end{tabular}
\end{enumerate}


\subsubsection{Data-set Route: /datasets}
The datasets route is meant for extending and querying information about the data-set which is used to train classifiers. 

\begin{enumerate}
    \item [\textbf{/append}]
\end{enumerate}







\subsubsection{Documents Route}
\subsubsection{Classifier Route}
\subsubsection{Indices Route}
\subsubsection{Indices Route}