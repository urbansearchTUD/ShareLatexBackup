\subsection{Selecting relevant data}
Because not all data from \maybe{information sources such as} CommonCrawl is relevant, we can make a selection of data. We can do this by only selecting the data that mentions at least two different cities. Making use of the comparative analysis of Rasool et al. \cite{rasool2012string} we chose the Aho-Corasick algorithm \cite{Aho-Corasick}, which is a multi-pattern exact string matching algorithm. We chose this because it is a fast exact string algorithm for which a well documented and maintained python library exists. This python library is called pyahocorasick \footnote{\url{https://pypi.python.org/pypi/pyahocorasick/}}  and is a fast and memory efficient python implementation of the Aho-Corasick algorithm. Using this python library we will reduce the amount of documents by disregarding the documents that don't mention at least two different cities. This reduces the amount of data that we need to parse in the next step of the pipeline.

We have chosen this approach because storing and indexing all documents/pages is not feasible as this requires large data storage. Because we don't have access to a fast and large data storage platform we will not store and index everything and then delete irrelevant documents.
