\section{Visualising the Data}

This section focuses on the visual representation of the processed data. This should be done in way that is easy to comprehend for users and helps them to interpret the data. To reach these goals, we identified the clients needs and desires. We have discussed the preferences of the client and have drawn up a global plan, which we present below.

\subsection{Representing the Data Graphically}

Since we are dealing with strongly related data, it is a natural choice to represent the data as a graph. We chose, in association with the client, to show the nodes and relations on a geographical map. Visualising cities on a map is intuitive to the user and we believe this will increase the ability of users to interpret the information in a productive manner.

\subsection{Using Geographical Maps}

We investigated two map libraries we can use to display our data on a map. The first one is Google Maps, which can be used freely and offers a lot of customisation options. The API is well defined and some of the group members have previously worked with it. The second option we investigated is Leaflet. Leaflet is an open-source JavaScript library that provides responsive maps. It also has fine grained API and lots of plugins available.
Both libraries are well suited for our needs. However, we decided to go with Google Maps, because of the existing experience of the group members. Another reason to gi with Google Maps is the amount of community support. This reasoning is best supported by the fact that querying "Google Maps" on StackOverflow.com returns 100.000+ results, while querying "Leaflet" gives us around 13000 results.

\subsection{Handling Map Clutter}

One of the challenges of visualising networks, as stated in \cite{468391}, is the occurrence of so-called map clutter. Map clutter means the network is displayed as an incomprehensible set of nodes and edges.
Several methods to prevent this are given in \cite{468391}. We will adopt some of these methods in our application, as explained next.

Users should be able to select what information they want to display. This will be included in the system by allowing the user to select cities and relations, enabling them to filter nodes and edges. The use of different sizes for nodes and edges or other attributes that are displayed can convey extra information to the user. We will use this to represent, for example, city population and exact strengths of relations.
The use of colour is another method mentioned in \cite{468391}. We will use colours to represent different types of relations and utilise colour intensity and opacity to represent the strengths of these different types of relations.







\begin{comment}
\subsection{Visualising the Data}
This subsection focuses on the visual representation of the processed data. This should be done in way that is easy to comprehend for users and helps them to interpret the data. To reach these goals, we identified the clients needs and desires. We have discussed the preferences of the client and have drawn up a global plan, which we present below.

\subsubsection{Representing the Data Graphically}

Since we are dealing with strongly related data, it is a natural choice to represent the data as a graph. The choice was made, in association with the client, to show the nodes and relations on a geographical map. Visualising cities on a map is intuitive to the user and we think this will increase the ability of users to interpret the information in a productive manner.

\subsubsection{Using Geographical Maps}

We investigated two map libraries we can use to display our data on a map. The first one is Google Maps, which can be used freely and offers a lot of customisation options. The API is well defined and some of the group members have previously worked with it. The second option we investigated is Leaflet. Leaflet is an open-source JavaScript library that provides responsive maps. It also has fine grained API and lots of plugins available.
Both libraries are well suited for our needs. However, we decided to go with Google Maps, because of the existing experience of the group members. Also, we feel Google Maps is better supported through documentation and communities.

\subsubsection{Handling Map Clutter}

One of the challenges of visualising networks, as stated in \cite{468391}, is the occurrence of so-called map clutter. Map clutter means the network is displayed as an incomprehensible set of nodes and edges.
Several methods to prevent this are given in \cite{468391}. We will adopt some of these methods in our application, as explained next.

Users should be able to select what information they want to display. This will be included in the system by allowing the user to select cities and relations, enabling them to filter nodes and edges. The use of different sizes for nodes and edges or other attributes that are displayed can convey extra information to the user. We will use this to represent, for example, city population and exact strengths of relations.
The use of colour is another method mentioned in \cite{468391}. We will use colours to represent different types of relations and utilise colour intensity and opacity to represent the strengths of these different types of relations.
\end{comment}