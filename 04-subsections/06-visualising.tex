\subsection{Visualising the Data}

This section focuses on the visual representation of the processed data. Our goals are to present the data and the things we learned during the processing of the data in a way that is easy to comprehend for users and can help ease the interpretation of the data.
To reach these goals we focused on the clients needs and desires. We discussed the preferences off the client and drew up a global plan, which we present below. We left space 

\subsubsection{Representing the data visually}

Since we are dealing with strongly related data, which is comprised off cities and the relations between cities, it was a natural choice to represent the data as a graph.
The choice was made, together with the client, to show the nodes and relations on a map. This was done because people are used to cities being visualised on a map and we think this will increase the ability of users to interpret the information in a productive manner.

\subsubsection{Maps}

We investigated two technologies we could use for the map on which we will display our data. The first one is Google Maps (GM). GM can be used freely and offers a lot of customisation options. The API is well defined and some of the group members worked with GM before. The second option we investigated was Leaflet. Leaflet is an open-source javascript library which provides responsive maps. It also has an well defined API and a lot of plugins.
Both libraries are well suited for our needs. 
We decided to go with GM, because of the experience of the group members with using GM. Also we feel that there are more resources available on GM, which would help us if we get stuck with an issue.

\subsubsection{Map clutter}

One of the challenges of visualising networks, as stated in \cite{468391}, is the risk of so called map clutter. This means the network is displayed as an incomprehensible set of lines and nodes.
Several methods to prevent this are given in \cite{468391}. We will adopt some of these methods in our application. The methods and use of these methods in our application are presented below.
Users should be able to select what information they want to display. This will be adopted in our application by allowing the user to (de)select cities and relations, which then will be shown/hidden. The use of different sizes for nodes and edges or other attributes that are displayed can convey extra information to the user. We will use this to represent sizes of cities (population) and strengths of relations.
The use of colour is another method which is presented in \cite{468391}. We will use colours to represent different types of relations and we will use intensity/opacity to represent the strengths of these different types of relations.