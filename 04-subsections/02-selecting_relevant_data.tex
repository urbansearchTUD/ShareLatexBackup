\subsection{Selecting Relevant Documents}
Because not all data from information sources such as Common Crawl is relevant to find relationships between cities, the data needs to be filtered. One way to do this, is to only select the data that mentions at least two different cities. Because the data is plain text, we need a way to scan through the text and determine if the text indeed has a co-occurrence of two different cities.
Making use of the comparative analysis of Rasool et al. \cite{rasool2012string}, we chose the Aho-Corasick algorithm \cite{Aho-Corasick}, which is a multi-pattern exact string matching algorithm. Using this algorithm, a predefined list of cities can be matched against the text of a web page or document. If at least two cities from the list appear in the text, we mark it as a useful document.

The decision to use the Aho-Corasick algorithm is strengthened by the fact that a well documented and stable Python library exists, which implements the aforementioned algorithm. This library is called \texttt{pyahocorasick}\footnote{\url{https://pypi.python.org/pypi/pyahocorasick/}} and is a fast and memory efficient implementation of the Aho-Corasick algorithm.

We make a selection of documents without storing the documents first, because storing and indexing all documents is not feasible due to storage constraints. As we do not have access to a fast and large data storage platform, we will not store and index everything first and then delete irrelevant documents.
