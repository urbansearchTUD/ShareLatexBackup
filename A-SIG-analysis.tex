\newpage
\section{Sig Feedback}\label{sig_fb}
\subsection{week 5}
[Analyse]

De code van het systeem scoort 4 sterren op ons onderhoudbaarheidsmodel, wat betekent dat de code bovengemiddeld onderhoudbaar is. De hoogste score is niet behaald door een lagere score voor Unit Complexity.

Voor Unit Complexity wordt er gekeken naar het percentage code dat bovengemiddeld complex is. Het opsplitsen van dit soort methodes in kleinere stukken zorgt ervoor dat elk onderdeel makkelijker te begrijpen, makkelijker te testen is en daardoor eenvoudiger te onderhouden wordt.

Omdat jullie qua score al vrij hoog zitten gaat het hier voornamelijk om kleine refactorings. Methodes als IndicesSelector.run\_workers en CoOccurrenceChecker.\_calculate\_occurrences zou je nog iets verder kunnen opsplitsen in functionele gebieden.

De aanwezigheid van test-code is in ieder geval veelbelovend, hopelijk zal het volume van de test-code ook groeien op het moment dat er nieuwe functionaliteit toegevoegd wordt.

Over het algemeen scoort de code bovengemiddeld, hopelijk lukt het om dit niveau te behouden tijdens de rest van de ontwikkelfase.

\subsection{week 9}