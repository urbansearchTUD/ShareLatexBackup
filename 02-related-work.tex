\section{Related Work}
In this section, we discuss several existing methods to measure relations and their strength between cities. Currently there are many different methods available to measure the strength of relations between cities, however these methods all have limitations. Below we will discuss the most common and why this does not give a good result for our problem.\\

One of the most common methods used is the interlocking network model (INM)\cite{taylor2012interlocking}. This model assumes cities have a flow of knowledge connection if there are offices of the same company in those cities. The biggest problem with this is that it is very limited. It only includes one relation type and it is discutable wether this is a good measurement for the relation. \\

When looking at digital data there are two different approaches: the cyberspace and the cyberplace \cite{devriendt2008cyberplace}. The cyberplace measures relations by using the infrastructure of the internet, for example internet hyperlinks, the structure of search engines and email traffic.\\
The cyberspace method focuses on the virtual communication of people through connected devices. One approach is by registering and mapping the number of pages indexed by search engines for queries containing the names of two cities. The problem is that large scale applications for this approach have been absent. The splitting of the relations between cities to different topics, for example, is missing. \\

Therefor we intend develop a computer program which does this. It will automatically find predefined relations between cities and their strength by using all pages available from search engines. In the following chapter we will investigate the requirements such a program should have.