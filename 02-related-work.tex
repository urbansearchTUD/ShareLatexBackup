\section{Related Work}
Since the 1960's, the desire to understand the modernisation of the economy, as seen by the increasing concentration of jobs and the cooperation between remote firms resulted in a surge of work on intercity relationships. \cite{tornqvist1968flows}.
One of the most common methods used is the interlocking network model (INM)\cite{taylor2012interlocking}. This model assumes cities have a flow of knowledge connection if there are offices of the same company in those cities. The biggest problem with this is that it is very limited. It only includes one relation type and it is disputable whether this is a good measurement for the relation \cite{lambregts2008geographies} because the question remains how much these offices are used for the exchange of knowledge and what kind of knowledge ares exchanged.

The last ten years there has been a lot of development in the field of data production and processing. Information retrieved from existing technologies which have made the automatic extraction of information and labelling a normality, could have an important role in understanding interurban relationships.

When looking at digital data there are two different approaches for determining intercity relationships: the cyberspace and the cyberplace \cite{devriendt2008cyberplace}. The cyberplace measures relations by using the infrastructure of the internet. Most research on this has been done on the 'backbone' of the internet made of cables and routers \cite{choi2006comparing, gorman2000networks}. \\

The cyberspace method focuses on the virtual communication of people through connected devices. One approach is by registering and mapping the number of pages indexed by search engines for queries containing the names of two cities\cite{devriendt2008cyberplace, janc2015visibility, janc2015geography}. In 2010 Brunn et al. evaluated the linkage between two cities by entering those cities into a search query followed by key words such as "global financial crisis" or "climate change" and registering the number of pages indexed \cite{brunn2010networks}. However, this method is very limited since you would have to manually enter a new query for each pair of cities for each relation. \\

To improve the textual analysis on websites and search engine queries to find digital links between cities a more systematic approach is needed.  A piece of software designed specifically for this purpose should automatically find predefined relations between cities and their strength by using all pages available from search engines or corpora. In the following chapter we will investigate the requirements for such a program.
