\section{Related Work}

In economics there is the question "What factors play a role in economic growth?". To answer this question you would first need to give a clear definition of economic growth itself. Economic growth can be seen as a positive change in the level of goods and services produced by a city over a certain period of time. An important characteristic is that economic growth is not the same in different sectors. Economic growth can be achieved when the rate of increase in total output is greater than the rate of increase in population of a city. \\

According to Harvard University three general theories of Economic Growth within cities \cite{glaeser1992growth} are those of Marshall-Arrow-Romer (MAR) theory (1986), Porter's theory (1990) and Jacobs theory (1969). These theories focus on knowledge spillovers and claim they are most effective in cities because communication between people is more extensive. The thoeries are based on in one company improves techniquely other companies near it will also benefit. These theories differ on whether monopolies or competition benifit the growth and whether the influence is within the same industry or not (e.g. brassiere and lingerie industry). \\
Other studies focus instead on the growth of countries instead of cities. Economist Alexander Cairncross wrote that the most important factors are investment, technical process, development and trade \cite{cairncross2013factors}. Economist Stanley Fischer focusses on the influence of macroeconomics (inflation, large budged deficits, distorted foreign echange markets) \cite{fischer1993role}. Economists Rudiger Dornbusch and Alejandro Reynoso claim the most important aspects differ per region \cite{dornbusch1989financial}. \\
In other words, there are many theories and there is much research on what plays an important factor in economic growth. Although MAR's Porter's and Jacob's theory do claim one of the reason for more economic growth in cities is due to more communication between people, much research into the connectivity between cities seem to be missing. One approach that has been taken is to look at where international companies are located. This only gives limited information however. Therefore we would like to see what information can be gathered from the internet by using search engines with input consisting of 2 cities.