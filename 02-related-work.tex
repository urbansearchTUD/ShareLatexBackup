\section{Related Work}
In this section, we discuss several existing methods to measure relations and their strength between cities. Currently there are only two methods to be considered. \\
%The first is to manually process data from search engines. The second is to look at what cities companies are located in. After discussing these two methods we will also take a look at NLTK \cite{nlkt_stemming}. NLTK is  a text processing tool on which (a part of) our system will be based.\\

\todo{
The most famous example of heuristic used in the research on world cities is the interlocking network model (INM) developed by Taylor (2001). This widespread model is based on the strong assumption that there are flows of knowledge and capitals between two cities if they host offices of the same firm. Apart from the fact that this assumption can be discussed (Lambregts, 2008), most of the applications of the INM deal with predefined samples of firms and cities, giving an incomplete assessment of the system. More generally, most of the studies on city networks use ‘stock’ variables rather than flows or links between cities.} \\

\todo{ 
Working on digital intercity linkages, Devriendt et al. (2008) distinguished two approaches to measure it – the cybersplace and the cyberspace – echoing the typology made by
Batty (1997). Firstly, the cyberplace approach investigates the linkages through the concrete infrastructure of the Internet. The bulk of efforts have concentrated on this “backbone” of Internet made of cables and routers (Choi et al., 2006; Gorman and Malecki, 2000). But it is well known that the infrastructures do not necessarily reflect the flows, especially in a digital context. In contrast, the cyberspace approach – which focuses on the virtual communications of people throughout connected devices – is much less developed. Existing attempts to measure digital interurban linkages in this way are still in their infancy. An interesting approach here is the registering and mapping of the number of pages indexed by search engines for a query containing two names of cities (Devriendt et al., 2008; Janc, 2015). However, large-scale applications of this approach have been absent and it requires further elaboration to truly judge its potential. For instance, content analysis of these linkages are still very rare and limited. An exception is Brunn et al. (2010) who registered the number of pages indexed for two cities followed by key words such as “global financial crisis” or “climate change” to evaluate the linkages between global cities concerning these two subjects. More advanced textual analysis on websites and search engine queries containing co-occurrences of city names would allow to better characterize digital links between cities in a more systematic approach. As there is no need for geographers to analyse the digital world in itself, the interest of this approach is to understand how the digital word reflects and influences the physical word. A comparative approach with traditional ways of measure interurban relation is then needed.} \\





The method that is currently used is looking at office locations of companies in different cities \cite{derudder2010intercity}. Although this might be a good indicator of the economic relations between cities, it completely ignores other relations cities might have with each other. According to Meijers, there might be many more factors \cite{meijers2007synergy}.\\

The aforementioned method is an incomplete measure of relationship strengths. It fails to look at the bulk of the factors (non-economical relationships) that play a role in forming relationships. \\

Another approach that has been considered would be to manually processing data from search engines. One queries a search engine (e.g. Google) checks how many websites are found. A small subset of the resulting pages is then classified manually to the relations. There are a few problems that arise with this approach. A lot of work needs to be done by hand, which takes a lot of man-hours. Furthermore, humans can be inconsistent when it comes to classifying documents. \\

Therefor we intend develop a computer program which does this work. It will automatically find predefined relations between cities and their strength by using all pages available from search engines. In the following chapter we will investigate the requirements such a program should have.