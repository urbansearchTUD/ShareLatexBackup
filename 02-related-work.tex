\section{Related Work}
In this section, we discuss existing methods to measure relations and their strength between cities. Currently there are only two methods to be considered. \\
%The first is to manually process data from search engines. The second is to look at what cities companies are located in. After discussing these two methods we will also take a look at NLTK \cite{nlkt_stemming}. NLTK is  a text processing tool on which (a part of) our system will be based.\\

The first method involves manually processing data from search engines. One queries a search engine (e.g. Google) checks how many websites are found. A small subset of the resulting pages is then classified manually to the relations. There are a few problems that arise with this approach. A lot of work needs to be done by hand, which takes a lot of man-hours. Furthermore, humans can be inconsistent when it comes to classifying documents, contrary to computers. \\

Another approach that is currently used is looking at office locations of companies in different cities \cite{derudder2010intercity}. Although this might be a good indicator of the economic relations between cities, it completely ignores other relations cities might have with each other. According to Meijers, there might be many more factors \cite{meijers2007synergy}.\\

The aforementioned methods are incomplete measures of relationship strengths. They fail to look at the bulk of the factors that play a role in forming relationships and scale badly. The framework described in this report aims to be a more robust solution for calculating intercity relationship strengths.
