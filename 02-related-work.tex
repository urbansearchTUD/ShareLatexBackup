\section{Related Work}
In this section, we discuss existing methods to measure relations and their strength between cities. Currently there are only two methods to be considered. \\
%The first is to manually process data from search engines. The second is to look at what cities companies are located in. After discussing these two methods we will also take a look at NLTK \cite{nlkt_stemming}. NLTK is  a text processing tool on which (a part of) our system will be based.\\

The method that is currently used is looking at office locations of companies in different cities \cite{derudder2010intercity}. Although this might be a good indicator of the economic relations between cities, it completely ignores other relations cities might have with each other. According to Meijers, there might be many more factors \cite{meijers2007synergy}.\\

The aforementioned method is an incomplete measure of relationship strengths. It fails to look at the bulk of the factors (non-economical relationships) that play a role in forming relationships. \\

Another approach that has been considered would be to manually processing data from search engines. One queries a search engine (e.g. Google) checks how many websites are found. A small subset of the resulting pages is then classified manually to the relations. There are a few problems that arise with this approach. A lot of work needs to be done by hand, which takes a lot of man-hours. Furthermore, humans can be inconsistent when it comes to classifying documents. \\

Therefor we intend develop a computer program which does this work. It will automatically find predefined relations between cities and their strength by using all pages available from search engines. In the following chapter we will investigate the requirements such a program should have.