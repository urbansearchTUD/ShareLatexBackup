\section{Related Work}
In the 1960s, progress in the knowledge of urban networks was made feasible by the rapidly increasing availability of useful data \cite{tornqvist1968flows}. \\
One of the most common methods used is the interlocking network model (INM)\cite{taylor2012interlocking}. This model assumes cities have a flow of knowledge connection if there are offices of the same company in those cities. The biggest problem with this is that it is very limited. It only includes one relation type and it is disputable whether this is a good measurement for the relation. \\

Since one decade, we are facing a new revolution in data production and processing. Developing innovative methods to extract relevant information from Big Data could again allow important progress in the knowledge of the interurban relationships. \\

When looking at digital data there are two different approaches: the cyberspace and the cyberplace \cite{devriendt2008cyberplace}. The cyberplace measures relations by using the infrastructure of the internet, for example internet hyperlinks, the structure of search engines and email traffic. Most research on this has been done on the 'backbone' of the internet made of cables and routers \cite{choi2006comparing, gorman2000networks}. \\

The cyberspace method focuses on the virtual communication of people through connected devices. One approach is by registering and mapping the number of pages indexed by search engines for queries containing the names of two cities\cite{devriendt2008cyberplace, janc2015visibility, janc2015geography}. 
\todo{In 2010 Brunn et al. registered the number of pages indexed for two cities followed by key words such as “global financial crisis” or “climate change” to evaluate the linkages between global cities concerning these two subjects. \cite{brunn2010networks}
More advanced textual analysis on websites and search engine queries containing co-occurrences of city names would allow to better characterize digital links between cities in a more systematic approach. } \\

Therefore we intend develop a computer program which does this. It will automatically find predefined relations between cities and their strength by using all pages available from search engines. In the following chapter we will investigate the requirements such a program should have.

