\chapter{Discussion}
% See http://libguides.usc.edu/writingguide/discussion for how to write a discussion

This section is divided into 4 parts. First we will answer the research question and comment on the sub-problems defined in the problem definition. Afterwards we will discuss the influence of these answers. Next we will mention issues we faced and which still remain. The last part of this section is dedicated to the ethical questions this project may involve.


\section{Answering the research question}

\todo{}

As mentioned in section \ref{sec:problem-definition-analysis}, the problem was the following:
\begin{quote} 
\centering 
\textit{How can open data be leveraged such that a metric for the strength of relationships between cities can be defined and visualised?}
\end{quote}

And the sub-problems were:
\begin{itemize}
    \item How can we filter the available text data to find co-occurrences of cities and discarding text data that does not contain co-occurrences? \\
    
    This should reduce the amount of data and thereby potentially speed up the rest of process.
    \item The sub-problem that arises after filtering is how to determine what relationships can be extracted from the text-data, this will be referred to as the classification of the text-data. \\
    
    This requires a method that reliably and efficiently processes the text-data and can be tuned to the clients wishes, meaning that the classification should output what the client desires. 
    \item The next question is how to store the data and determine the strength of the relationships. 
    \item The last question is how to combine the stored data and present it to a user, this means visualising and/or exporting the data in an accessible way.
\end{itemize}

\section{influence of the answers}
\todo{}

\section{issues faced and remaining}
\todo{}

% describe issues faced during implementation, as well as issues that are still 
% open. Main thing here is to discuss the Neo4j python driver with multiprocessing,
% which we could not get to work. Also mention the state of the document 
% classification

\section{Ethics}
When using our program, there are two ethical issues that may arrive. The first is due to the fact that pages may be downloaded and stored, which may result in privacy or copyright issues. The second is about what the results of our program may be used for, and how the world will react to this.

\subsubsection{Storage}
One ethical issue is due to the storing of data. Since we store random web pages we do not know whether or not these pages may contain private or copyrighted data. For instance news articles could be downloaded and stored whilst this could violate the copyright issues. For most free news sites this is not a problem but this especially becomes a problem when using Delpher as a data source. Therefor if this source is added it must also be ensured that the data is stored in a safe way.

\subsubsection{Influence}
Another issue that may occur is the influence this kind of research may have. If there is indeed a correlation between the results of our program and the economic growth of cities this may influence the behaviour of investors, companies and cities. Investors may look at the data and decide to invest in companies from more growing cities. Companies may use this data to decide where to build their new offices. And cities may change their policies based on the results. In future executions of our system it may also occur that data is being manipulated. Involved parties which put extra data online containing the names of cities they want to have a better result for. Whilst these issues may occur, we do not suspect our system to have a large enough impact to cause this. It may rather be a step towards these effects. Over time the effects will become clearer and they should be taken into account when continuing research in this field.
