\chapter{Conclusion}
In the past few months we worked towards building an application to find relationships between cities by using data from online sources.

First, in section \ref{sec:related work} we discovered that the related work currently present is either very limited or questionable. Next, in section \ref{sec:problem-definition-analysis} we identified the requirements for a solution to the problem and discuss issues that might arise. Afterwards in sections \ref{sec:framework-and-tools} and \ref{sec:framework-implementation} we described a framework that satisfies the requirements and tackles the issues and the implementation of this framework. In section \ref{sec:project-evaluation} we discussed the fulfilment of the requirements and the design goals, and evaluated the process. Next, in section \ref{sec:discussion} we discussed the the issues we are still facing, the results of the classification, and the ethical issues our project might induce. Last, in section \ref{sec:recommendations} we made recommendations for future projects on this subject.

As explained in section \ref{sec:project-evaluation} all of the design goals, except the exporting of extracted relations, were met. While there are still some issues, the most important requirements were also successfully implemented.

We can now reflect on the problem definition from \ref{sec:3-problem-definition}: how can open data be leveraged such that a metric for the strength of relationships between cities can be defined and visualised?

One way open data can be leveraged such that a metric for the strength of relationships between cities can be defined and visualise is as follows. First, text data is collected from a corpus (in our case documents from Common Crawl). Each document is then checked for the occurrence of two or more city names by using the multi-pattern string matching and is discarded if it does not meet this check. This selection of documents is classified according to predefined relationship types using the SVM machine learning algorithm. The documents are stored on disk and the relations are stored in the graph database Neo4J. The strength of relationships between two cities is then found by counting the number of all documents for each relationship that contain the two city names. This is visualised using a Web application.

With this, we believe we have proven the application can be an asset to the research on intercity relations and we hope this will give our clients a tool to help with their further research. 
