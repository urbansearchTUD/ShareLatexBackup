\chapter{Conclusion}
In the past few months we worked towards building an application to find relationships between cities by using data from online sources.  \\

First, in section \ref{sec:related work} we discovered that the related work currently present is either very limited or questionable. Next, in section \ref{sec:problem-definition-analysis} we identified the requirements for a solution to the problem and discuss issues that might arise. Afterwards in sections \ref{sec:framework-and-tools} and \ref{sec:framework-implementation} we described a framework that satisfies the requirements and tackles the issues and the implementation of this framework. In section \ref{sec:project-evaluation} we discussed the fulfilment of the requirements and the design goals, and evaluated the process. Next, in section \ref{sec:discussion} we discussed the the issues we are still facing, the results of the classification, and the ethical issues our project might induce. And last, in section \ref{sec:recommendations} we made recommendations for future projects on this subject. \\

As explained in section \ref{sec:project-evaluation} all of the design goals except the exporting of data were met. Whilst there are still some issues, the most important requirements also passed. From the sub problems in section \ref{sec:3-problem-analysis}, filtering was discussed in \ref{sec:5-filtering}, classification in \ref{5-classification}, storing data in \ref{sec: 5-storing} and data visualisation \& export in \ref{sec:5-front-end}. We can now reflect on the problem definition from \ref{sec:3-problem-definition}: "How can open data be leveraged such that a metric for the strength of relationships between cities can be defined and visualised?" \\

To which the answer is: One way open data can be leveraged such that a metric for the strength of relationships between cities can be defined and visualise is by first downloading text data from a storage (in our case documents from CommonCrawl). Each document is then checked for containing two or more city names by using the Pyahocorasick package and is discarded if it does not meet the check. This selection of documents is classified according to predefined relationships between cities using the SVM machine learning algorithm. The documents are stored on the disk and the relations are stored in the visual graph database Neo4J. The strength of each relationships between two cities is then found by counting the number of all documents for each relationship that contain the two city names. This is visualised by NodeJS. \\

With this, we believe we have proven the application can be an asset to the research on factors leading to economic growth and we hope this will give our clients a tool to help with their further research. 

