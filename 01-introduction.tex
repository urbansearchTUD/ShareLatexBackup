\section{Introduction}
With the development of future cities in mind, the interest in city networks has grown over the years. However, appropriate information on how cities are connected and the strength of these connections is hard to find. According to Short et al. comparative statistics are not easily available and common assertions are repeated \cite{short1996dirty}. Although more research was published since then, for example the work of DeRudder et al. \cite{derudder2005appraisal}, there is still a lot of information missing on relationships between cities.
 In order to get more insight into these relationships, there is a need for complex data analysis. \\
%However, it is still unknown what primarily causes economic growth. It could either be these benefits or an increase in the contesting position of the agglomeration in the global urban network. \\

The huge amount of textual data generated online and the numerous historic archives, such as for example Delpher\footnote{\url{http://www.delpher.nl}} and the British Newspaper Archive\footnote{\url{http://www.britishnewspaperarchive.co.uk/}}, are great sources of information on social and economic behaviours. The clients hypothesis is that advanced text mining on newspapers and web pages containing city names would allow for a better understanding of the role of information in shaping urban systems. Similar to research efforts in other domains, such as financial trade \cite{preis2013quantifying} and sales forecasting \cite{wu2014future}, the clients wish is to develop an application that captures urban-urban interactions. These interactions are retrieved from information corpora through the co-occurrence of geographical names in textual data. An example of how one could try to achieve this using  the Google search engine\footnote{\url{https://www.google.com}} is \texttt{"Rotterdam + Amsterdam " OR "Amsterdam + Rotterdam"}, which searches for the co-occurrence of Amsterdam and Rotterdam. However, manually processing all results a search engine yields is not feasible because someone would have to read each page to determine which types of relationships the page contains. An application should process all the pages that contain co-occurrences of cities to determine what type of relations, for example transportation or leisure, between cities can be extracted from the document. Thus, we will answer the following question: 
how can the strength of relationships between cities be extracted and visualised from open data using software? \\
%what approach is best suited for extracting and visualising the strength of relationships between cities through open data?
%what approach is best suited for extracting and visualising the global urban network from open data? \\

First, we discuss related work in section 2. Second, we identify the requirements for a solution to the problem and discuss issues that might arise in section 3. Third, we develop a methodology for a framework that satisfies the requirements and tackles the issues in section 4. In the fifth section we discuss evaluating the system. We conclude in section 6 with the results of our research.