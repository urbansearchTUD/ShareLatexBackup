\newpage
\section{Introduction}

% Nowadays, many people live in or around an agglomeration, to reap the benefits it comes with. An agglomeration in this sense is an extended town area consisting of the built-up area of a central place and any suburbs linked by continuous urban area. A universally accepted idea is that benefits of agglomerations are key to economic growth \cite{porter2000location}. In order to get more insight in the relationships between cities, there is a need for complex data analysis. \\
%However, it is still unknown what primarily causes economic growth. It could either be these benefits or an increase in the contesting position of the agglomeration in the global urban network. \\

The huge amount of textual data generated online and the numerous historic archives are great sources of information on social and economic behaviours. Advanced text mining on newspapers and web pages containing city names would allow for a better understanding of the role of information in shaping urban systems. Similar to research efforts in other domains, such as financial trade \cite{preis2013quantifying} and sales forecasting \cite{wu2014future}, the idea is to develop search queries that capture urban-urban interactions. These interactions are retrieved from information corpora through the co-occurrence of geographical names in textual data. An example of such a query on the Google search engine\footnote{\url{https://www.google.com}} is \texttt{"Rotterdam + Amsterdam " OR "Amsterdam + Rotterdam"}, which searches for the co-occurrence of Amsterdam and Rotterdam. However, manually processing all results a search engine yields is not feasible. Thus, we will answer the following question: 
how can the strength of relationships between cities be extracted and visualised from open data? \\

%what approach is best suited for extracting and visualising the strength of relationships between cities through open data?
%what approach is best suited for extracting and visualising the global urban network from open data? \\

First, we discuss related work in section 2. Second, we identify the requirements for a solution to the problem and discuss issues that might arise in section 3. Third, we develop a methodology for a framework that satisfies the requirements and tackles the issues in section 4. We conclude in section 5 with the results of our research.