\providecommand{\keywords}[1]{\textbf{Keywords:} #1}

\begin{abstract}

%It is yet to be discovered how the importance of cities in the global network can be elucidated. In this document, the requirements for a solution to this problem are identified, as well as possible issues that might arise. We explore various methods and select the most appropriate ones that, if combined, satisfy these requirements and tackle the identified issues, in preparation for an attempt at solving the problem.\\

It is hard if not impossible to measure the strength of relationships between cities using existing technologies. Because of this, it remains uncertain how exactly economic growth is affected by urbanisation. The universally accepted explanation is that only increasing size of cities affects economic growth \cite{porter2000location}. In this report, we develop a methodology that allows for determining intercity relationship strengths, using open data. For this, we evaluate whether graph databases like Neo4j \cite{neo4j} or document search engines such as ElasticSearch \cite{elasticsearch} are best suited and describe machine learning algorithms for categorising data based on the co-occurrence of city names. Additionally, we present visualisation techniques to be able to intuitively analyse the results. \\

%It is yet to be discovered how the strength of relationships between cities relates to economic growth. Many believe it is merely the increasing size of cities that cause economic growth. In this report, we develop a methodology that allows for determining intercity relationship strengths, using open data. For this, we evaluate whether graph databases like Neo4j\cite{neo4j} or document search engines such as ElasticSearch\cite{elasticsearch} are best suited and describe machine learning algorithms for categorising data based on the co-occurrence of city names. Additionally, we present visualisation techniques to be able to intuitively analyse the results. \\

\noindent\keywords{urban, relation strength, data mining, graph, classification, filtering}
\end{abstract}