\section{Introduction}


%mooier opschrijven
Common belief is that agglomeration benefits are key to economic growth \cite{porter2000location}. 
However it may be that this econmic growth's primary cause is the increase in (inter)national network embeddedness. \\

The huge amount of textual data generated online or the numeric historic archives are great sources of information on social and economic behaviours and constitute the bulk of information flowing among each other. Advanced text mining on newspapers and websites containing city names would allow to better understand the role of information in shaping urban systems. \\
Similar to research efforts in other domains such as financial trade \cite{preis2013quantifying} and sales forecasting \cite{wu2014future}, the idea is to develop search queries that capture urban-urban interactions as they can be found on the web through the co-occurence of geographical names on websites e.g. "Zeeuws-Vlaanderen + Amsterdam " OR "Amsterdam + Zeeuws-Vlaanderen". \\

We will start by looking at related work that has been done on data from search queries, discussing the similarities to our research. Next we will further analyse the problem in its four mian subsections: the extraction of data from the internet; the filtering and categorizing of this data; the development of search-able queries and last the visualization of the found data. From this problem analysis we will draw the requirement analysis. The requirement analysis is then used to decide upon the framework and tools we are going to use. And last we will give a short conclusion.