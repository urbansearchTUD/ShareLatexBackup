\chapter{Project Evaluation}
\todo{intro}
% evaluate the product, and see whether we have at least the minimum viable product
% using the requirements and design goals (possibly in a table)
% make sure to add lots and lots of figures, tables and benchmarks here
% also add SIG stuff and talk about SCRUM

\todo{Put results in appendix F}
\todo{write conclusion about results}

\section{Evaluation of Requirements}
In section \ref{sec:reqs} we declared the requirements for our program. Table \ref{requirements_pass/fail} shows which of these requirements passed or failed and why. Failed requirements are discussed in section \ref{sec:Discussion - Open Issues}. As can be seen most of the requirements passed, but there are also some that failed. Most of the failed requirements are acceptable, but the failed must have is planned to be fixed in the last week of coding. Since this is after the due date for this report this fix can not be included in the report.\\

\begin{table}[]
\centering
\caption{My caption}
\label{requirements_pass/fail}
\begin{tabular}{ll m{8cm}}
\textbf{Must Haves}                     & Pass / Fail & Comment                                                                                                                                                                                                                               \\
1 Mining from Common Crawl     & Pass        & Data is successfully gathered from Common Crawl.                                                                                                                                                                                       \\
2 Exporting relations          & Fail        & Due to many delays in other parts, exporting the data was not achievable within the time set for the product.                                                                                                                         \\
3 Extracting relations         & Pass        & Relations are successfully extracted from documents for about \todo{x} \%.                                                                                                                                                            \\
4 Visualisation                & Pass        & A front-end which shows the data on a map is ready for use.                                                                                                                                                                           \\
5 Present statistics           & Pass        & The front-end presents statistics with the shown data.                                                                                                                                                                                \\
                               &             &                                                                                                                                                                                                                                       \\
\textbf{Should Haves}                   & Pass / Fail & Comment                                                                                                                                                                                                                               \\
1 Hierarchical relations       & Fail        & Since extracting relations from text documents was more challenging than first thought, hierarchical relations were not included.                                                         \\
2 Machine learning retrainable & Pass        & It is possible to retrain the machine learning by feeding it a set of labelled documents.                                                                                                                                              \\
3 Add large data sets           & Pass        & It is possible to add large  data sets, however since it does cause an increase in time needed to run the algorithm, we did not use a large data set for our demo version.                     \\
4 Duplicate city names         & Fail        & The algorithm does not take cities with duplicate names, or names fitting to multiple cities into account, due to time constraints.                                                         \\
                               &             &                                                                                                                                                                                                                                       \\
\textbf{Could haves}                    & Pass / Fail & Comment                                                                                                                                                                                                                               \\
1 Use Delpher                  & Fail        & It is not possible to use data from Delpher unless it is already downloaded and stored.                                                                                                                                               \\
2 Visualisation for comparing  & Pass        & The front-end includes visualisation for comparing cities and relationships to each other.                                                                                                                                            \\
                               &             &                                                                                                                                                                                                                                       \\
\textbf{Would likes}                    & Pass / Fail & Comment                                                                                                                                                                                                                               \\
1 Show all connections         & Fail        & Theoretically it is possible to do this, but it would result in a map on which relationships can not be differentiated from each other due to the large amount the system was not build for. \\
2 Other than .nl data          & Fail        & The classifier is only trained on Dutch domains.                                                                 
\end{tabular}
\end{table}

\section{Evaluation of Design goals}
In section \ref{sec:design-goals} we discussed the design goals for this project. We came up with seven design goals on which we will reflect.

\subsection{Credible}
\todo{after results}

\subsection{Understandable}
\todo{after results}

\subsection{Scalable}
For scalability the goal was to allow the project to be scalable so that websites from other domains than .nl can be used. Since we used Neo4J for storing relationships and Neo4J is highly scalable, storing the relationships for other domains than .nl should not be a problem. 

\subsection{Plugable}
The design goal plugable the goal was to make the application able to perform analysis on different data sets without the need of a developer. Whilst this is possible, the documents from other data sets would have to be in the correct format before they can be processed. For small data sets this might not be that much of a problem, but for larger data sets it would become very time-consuming without the help of a developer.

\subsection{Exportable}
At this moment we have not met the design goal exportable, which was to ensure the numeric data could be exported, for example in CSV format. Since this is not a vital part of the application, this has not yet been included. We are however still working on this. 

\subsection{Fast development}
Another goal was to have a fast development cycle because of the time constraints. To do this we choose tools, applications and programming languages with which members of our team were already experienced with. Even so, some setbacks occurred which caused the cycle to slow down.

\section{Product evaluation}
\todo{after results, check if this is right}
To conclude the two previous sections; we have a functioning program that consists of most vital components. Whilst the results from the application are not exportable (yet), the other design goals were met. The program provides credible and understandable which can be used to analyse relationships between cities in the Netherlands. It can also be extended for non-Dutch cities and pages. Most of the requirements also passed, but depending on the needs of the end-user, it would not be difficult to extend to program to include the others as well. Even though functionality might be missing this is a useful application for analysing data and as a demo it proves systems like these are worth wile to develop.

\section{Process}
In this section, we evaluate the development process and explain what methods were used and if they were used correctly. Additionally, we discuss the collaboration with the client and coach and within the group.

\subsection{Development Process Evaluation}
In order to have a smooth development cycle, we made several agreements in the beginning of the project. All code changes had to be submitted through a pull request and needed to be reviewed before they could be merged into the main code base. This was enforced using the project settings in GitHub, where branch access can be regulated. We believe that this approach has helped us greatly to write good quality code and to make sure everyone knew what was going on.

Furthermore, we used Travis CI\footnote{\url{https://travis-ci.org/}} for continuous integration. Building was automatically triggered by both pull requests and normal pushes. GitHub also provides an option to require status checks to pass before being able to merge. However, we disabled this due to some testing stability issues in the first few weeks of development. We did however agree to only merge when all status checks passed. In all but a few hasty merges we managed to adhere to this agreement. Because of the continuous integration of Travis CI, we hardly ever had to deal with unforeseen integration problems of new features.

Travis was also configured to submit coverage reports to Coveralls\footnote{\url{https://coveralls.io}}, so we could easily monitor how test coverage was affected by code changes. Rule of thumb was that all newly added files should have at least 80\% code coverage. Through Coveralls, we could quickly verify this. \todo{add final test code \#lines, coverage \%, etc.}

The product development was managed using the agile development methodology Scrum. Each week was a single sprint. We kept track of the current sprint and the product backlog in Trello\footnote{\url{https://trello.com}}, which helped us to have a clear overview of the product's status. We did however notice that weekly sprints were a bit too short. Usually, a sprint was too full and in the end we noticed that we got somewhat careless about sprints. Moreover, at times, unexpected time consuming issues lead to not finishing the sprint at all. It might therefore be beneficial to extend sprint duration to two weeks to allow for unexpected problems.

The system was initially run on a relatively small virtual private server (4GB RAM, 2 CPUs, 150GB SSD) of one of the group members. With increasing database size, we noticed that we would require more resources, especially RAM. We therefore asked the client to request a server of the TU Delft that we could use for the application. After a few weeks of inefficient (mis)communication, we eventually got access to a 8GB RAM, 4CPUs, 100GB HDD virtual machine. This server meets the minimal requirements for the system, but does not provide enough resources if the data set is extended to more than a million documents. Moreover, the virtualisation is not ideal for the many disk IO the application requires. Therefore, it would have been better to have a physical device at hand.