\section {Requirement Analysis}
In this section, we first define the design goals. Then, we list the requirements which the {\color{red} FIXME thingy} meet. We use the MoSCoW method\cite{clegg1994case} as a prioritisation technique. Four levels of priority are defined: must have, should have, could have and won't have (also known as would like).

\subsection{Design Goals}
The high-level design goals for this project have been provided by the client. These serve as a guideline to determine the priority label of the specific requirements, as defined in section \ref{sec:reqs} and are listed here, ordered by priority.

\paragraph{credible} The client wants to impugn a widely spread belief. Therefore, the basis on which he does that must be sound.
\paragraph{understandable} The results of the {\color{red} FIXME system?} should be visually understandable, so it is easy for the client to deduce conclusions from them. Additionally, retrievable numeric data enable the client to further investigate the results outside of the scope of the {\color{red} FIXME system?}, should the need arise.
\paragraph{scalable} The client has expressed his concerns that restricting the {\color{red} FIXME system?} to a set of non-English domains might impair the probability that his research will be published in an acknowledged journal. Allowing for investigating other domains would greatly help the client in a later stadium.
\paragraph{plugable} The client conveyed it might be interesting to let the {\color{red} FIXME system?} perform analysis on different data sets without the need of a developer. 

\subsection{Product Requirements}\label{sec:reqs}
\subsubsection {Must Have}
Requirements labelled as "must have" are key to the minimal performance of the {\color{red} FIXME product/whatever}. If they are not met, the {\color{red} FIXME: thingy} can be considered a failure.

\begin{enumerate}
    \item{General} 
    \begin{enumerate}
        \item Adding city names
        \item Grouping relations and “zooming” on these relations
    \end{enumerate}
    
    \item{Search Engine} 
    \begin{enumerate}
        \item Filter results
        \item Data mining
    \end{enumerate}
    
    \item{Filtering} 
    \begin{enumerate}
        \item Logic Filters 
        \item Relations Filters
    \end{enumerate}
    
    \item{Machine Learning} 
    \begin{enumerate}
        \item Types of relations
    \end{enumerate}
    
    \item{Visualisation}
    \begin{enumerate}
        \item Statistics of relations? Query relations
        \item Strength of relations
        \item Types: ML CBS defined
    \end{enumerate}
\end{enumerate}


\subsubsection {Should Have}
"Should have" requirements are those that greatly improve system performance and/or usability but might not fit in the available development time.

\begin{enumerate}
    \item{General}
    \begin{enumerate}
        \item Pluggable datasets
    \end{enumerate}
    
    \item{Machine Learning}  
    \begin{enumerate}
        \item Generalising relations, grouping relations
    \end{enumerate}
\end{enumerate}


\subsubsection {Could Have}
Requirements labelled as "could have" are useful and should be included in the system if time and resources permit.

\begin{enumerate}
    \item{General}
    \begin{enumerate}
        \item International city names
    \end{enumerate}
    
    \item{Visualisation}    
    \begin{enumerate}
        \item Front end for the app
    \end{enumerate}
\end{enumerate}

\subsubsection {Would Like}
"Would like" requirements have been agreed upon as not important to deliver within the current time schedule. However, they can be included in future releases.